\chapter{Samenvatting}
\ifpdf
    \graphicspath{{Summary/Figs/Raster/}{Summary/Figs/PDF/}{Summary/Figs/}}
\else
    \graphicspath{{Summary/Figs/Vector/}{Summary/Figs/}}
\fi

\begin{otherlanguage}{dutch}
\glsreset{MLA}
De populariteit van \glsfirst{p2p} toepassingen, en daarmee ook het P2P verkeer op het internet, is in de laatste jaren sterk toegenomen. 
Deze toename is naast het gebruik van goedaardige P2P toepassingen ook te wijten aan kwaadaardige P2P toepassingen zoals P2P \glspl{botnet}. 
Om de toenemende bedreigingen van P2P botnets te pareren, is actieve bestrijding ervan noodzakelijk. 
Een eerste stap daarin is om te detecteren welk internetverkeer deel uitmaakt van P2P botnets. 
In dit onderzoek is daarmee een start gemaakt door te kijken of internetverkeer geclassificeerd kan worden als P2P verkeer en niet-P2P verkeer, nog ongeacht of dat goed- of kwaadaardig verkeer betreft.\\

\noindent
Classificatie van P2P verkeer is uitdagend aangezien traditionele technieken, die hoofdzakelijk poortnummers of payload-informatie analyseren, ineffectief zijn tegen toepassingen die willekeurige poorten of encryptie gebruiken. 
In het onderzoek is, op basis van literatuuronderzoek, \gls{ML} gebruikt als methode voor classificatie van P2P verkeer, waarbij de algoritmen J48, REPTree en AdaBoost gebruikt zijn voor analyse van statistische flow features die zowel poort- als payload agnostisch zijn.\\

\noindent
Het classificatie mechanisme leert P2P gedrag van een data set die bestaat uit zowel goedaardig P2P-verkeer, kwaadaardig P2P-botnet verkeer en niet-P2P verkeer.
De nauwkeurigheid van de classifier op de daadwerkelijke test data bepaalt hoe effectief er onderscheid kan worden gemaakt tussen P2P en niet-P2P verkeer.
De performance metrieken van de classifier zijn allen gebaseerd op het gebruik van test data sets, waarbij elke individuele set disjunct is met de overige sets(inclusief de training set).
Uit de resultaten van dit kwantitatief empirisch onderzoek is gebleken dat hiermee een hoge nauwkeurigheid kan worden bereikt, die vergelijkbare bestaande benaderingen voor classificatie van P2P verkeer overtreft.\\


\noindent
De datasets en enkele broncodes die tijdens het onderzoek werden gebruikt zullen publiekelijk ter beschikking worden gesteld om bijvoorbeeld validatie of uitbreiding van dit werk mogelijk te maken.


\end{otherlanguage}


\vspace{2cm}

\noindent
\textbf{Trefwoord: P2P traffic, port agnostic, payload agnostic, classification, Machine learning}

\chapter{Summary}
\ifpdf
    \graphicspath{{Summary/Figs/Raster/}{Summary/Figs/PDF/}{Summary/Figs/}}
\else
    \graphicspath{{Summary/Figs/Vector/}{Summary/Figs/}}
\fi

\glsreset{MLA}
The popularity of \gls{p2p} applications, and consequently the P2P traffic on the internet, has increased in the last years.
This increase in traffic usage of P2P applications is besides benign P2P applications also due to malicious P2P software such as P2P botnets.
To cope with the increasing threats imposed by malicious P2P botnets, botnets should be combated actively. A first step is to  detect which internet traffic originates from P2P \glspl{botnet}.
In this research, a start has been made by looking at whether internet traffic can be classified as either P2P traffic or non-P2P traffic, yet regardless of whether it concerns benign or malicious traffic.\\

\noindent
Classification of P2P traffic is challenging since traditional techniques, that mainly analyze port numbers or payload data, are becoming ineffective against applications that use random ports or encryption.
This research proposes, based on literature study, \gls{ML} as a method for P2P traffic classification, using the algorithms J48, REPTree and AdaBoost for analysis of statistical flow features, which are both port and payload agnostic.\\

\noindent
The classifier is trained with a data set consisting of network traffic derived from four P2P applications, two P2P botnets and non-P2P traffic.
Classifier metrics were obtained by utilizing test data sets, in such a way that each individual set is disjunct with all the other sets(including training set). 
The results of this quantitative empirical research show that the proposed method can achieve high accuracy, outperforming comparable existing approaches for classification of P2P traffic.
\\

\noindent
The data sets and some source codes used in the thesis will be made available to the research
community to enable validation and extension of the work.

\vspace{2cm}

\noindent
\textbf{Keywords: P2P traffic, port agnostic, payload agnostic, classification, Machine learning}
